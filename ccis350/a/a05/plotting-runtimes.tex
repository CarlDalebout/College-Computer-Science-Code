\newpage\textsc{Plotting Runtimes}

Let me show you how to create a graph in a pdf using a library that I built.
Here's a graph:

\begin{python}
from latextool_basic import *
plot = FunctionPlot(width="3in", height="2in")
plot.add(((100, 1),
          (200, 2),
          (300, 0.5)),
          line_width='2', color='red', legend='bubblesort')
plot.add(((100, 5),
          (200, 3),
          (300, 7)),
          line_width='2', color='green', legend='bubblesort2')
plot.add(((100, 7),
          (200, 4),
          (300, 5)), line_width='2', color='blue', legend='selectionsort')
print(plot)
\end{python}

The above graph is built by this code:
\begin{console}[fontsize=\footnotesize]
\begin{python}
from latextool_basic import *
plot = FunctionPlot(width="3in", height="2in")

plot.add(((100, 1),
          (200, 2),
          (300, 0.5)),
         line_width='2', color='red', legend='Bubble sort')
         
plot.add(((100, 5),
          (200, 3),
          (300, 7)),
         line_width='2', color='green', legend='Insertion sort')
         
plot.add(((100, 7),
          (200, 4),
          (300, 5)),
         line_width='2', color='blue', legend='Selection sort')

print(plot)
\end{python}
\end{console}

In \verb!q07.tex!, modify the graph with your own data.
It should be clear where to put your data.
Look at the \verb!plot.add!.
\verb!plot.add! adds a graph:
\begin{Verbatim}[frame=single,fontsize=\footnotesize]
plot.add(((100, 1),
          (200, 2),
          (300, 0.5)),
         line_width='2', color='red', legend='Bubble sort')
\end{Verbatim}
The \verb!(100, 1)!, etc are the points on this graph.
Be careful you don't lose any parentheses or commas, or add any extras.
Then run \verb!make! and a file \verb!main.pdf! will be
generated -- your graph should appear in the pdf.

If you run \verb!make! and have problems, you must talk to
someone in CS hangout of in CCCS Discord right away --
this means that something is wrong with your
virtual machine.

The above graph covers an area of 3-by-2 inches.
You should probably use a larger graphing area.
You should increase the width and height of the graph.
Also, here are some options for color:
red,
green,
blue,
cyan,
yellow,
black,
pink,
brown,
teal.
